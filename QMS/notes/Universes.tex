%%
% On H-S universes
% April 2022
%%
%\documentclass[12pt]{amsart}
\documentclass[11pt]{article}

%\usepackage{ebgaramond}
%\usepackage{euscript}

\usepackage{amsmath}
\usepackage{amssymb,latexsym}
\usepackage{amsthm}
\usepackage{bm}
\usepackage{bbm}
\usepackage[all,cmtip]{xy}
\input{diagxy}
%\CompileMatrices       
\usepackage{url}
%\usepackage{fullpage}
\usepackage[colorlinks=true,linkcolor=blue,anchorcolor=blue,citecolor=blue,
     		filecolor=blue,urlcolor=blue]{hyperref}
\newcommand{\ednote}[1]{[\textit{\color{red}{#1}}]} % editorial note

% categories
\newcommand{\bbtwo}{\ensuremath{\mathbbm{2}}}
\newcommand{\C}{\ensuremath{\mathbb{C}}}
\newcommand{\A}{\ensuremath{\mathbb{A}}}
\newcommand{\N}{\ensuremath{\mathbb{N}}}
\newcommand{\D}{\ensuremath{\mathbb{D}}}
\newcommand{\B}{\ensuremath{\mathbb{B}}}
\newcommand{\T}{\ensuremath{\mathbb{T}}}
\newcommand{\CC}{\ensuremath{\mathcal{C}}}
\newcommand{\WW}{\ensuremath{\mathcal{W}}}
\newcommand{\FF}{\ensuremath{\mathcal{F}}}
\newcommand{\BB}{\ensuremath{\mathcal{B}}}
%\newcommand{\EE}{\ensuremath{\mathcal{E}}}
\newcommand{\op}[1]{\ensuremath{{#1}^{\mathrm{op}}}}
\newcommand{\pshx}[1]{\ensuremath{\mathsf{Set}^{\op{#1}}}}
\newcommand{\psh}[1]{\ensuremath{[\op{#1},\mathsf{Set}]}}
\newcommand{\Set}{\ensuremath{\mathsf{Set}}}
\newcommand{\Cat}{\ensuremath{\mathsf{Cat}}}
\newcommand{\covpsh}[1]{\ensuremath{\mathsf{Set}^{#1}}}
\newcommand{\cSet}{\ensuremath{\mathsf{cSet}}}
\newcommand{\pocorner}[1][dr]{\save*!/#1+1.2pc/#1:(1,-1)@^{|-}\restore}
\newcommand{\pbcorner}[1][dr]{\save*!/#1-1.2pc/#1:(-1,1)@^{|-}\restore}
%\newcommand{\pbcornertransp}[1][dr]{\save*!/#1+1000pc/#1:(5000,5000)@^{|-}\restore}
\newcommand{\y}{\ensuremath{\mathsf{y}}} % Yoneda embedding
\newcommand{\yon}{\ensuremath{\mathsf{y}}} % Yoneda embedding
\newcommand{\Hom}{\ensuremath{\mathrm{Hom}}}
\renewcommand{\hom}{\ensuremath{\mathrm{Hom}}}
\newcommand{\plusalg}[1]{\ensuremath{^+\!\mathsf{Alg}{(#1)}}}
\newcommand{\plusalgx}{\ensuremath{^+\!\mathsf{Alg}}}


% arrows
\newcommand{\hook}{\ensuremath{\hookrightarrow}}
\newcommand{\mono}{\ensuremath{\rightarrowtail}}
\newcommand{\ra}{\ensuremath{\rightarrow}}
\newcommand{\cof}{\ensuremath{\rightarrowtail}}
\newcommand{\fib}{\ensuremath{\epi}}
\renewcommand{\to}{\ensuremath{\rightarrow}}
\newcommand{\too}{\ensuremath{\longrightarrow}}


% cubical sets
\newcommand{\I}{\ensuremath{\mathrm{I}}}
\newcommand{\II}{\ensuremath{\mathbb{I}}}
\renewcommand{\H}{\ensuremath{\mathbb{H}}}
\newcommand{\HH}{\ensuremath{\mathcal{H}}}

% type theory
\newcommand{\G}{\ensuremath{\Gamma}}
\newcommand{\defeq}{=_{\mathrm{def}}}
\newcommand{\type}{\mathsf{type}}       
\newcommand{\types}[2]{#1 \vdash #2:\type}
\newcommand{\Gtypes}[1]{\types{\Gamma}{#1}}
\newcommand{\term}[2]{#1\,:\,#2}
\newcommand{\terms}[2]{#1 \vdash #2}
\newcommand{\Gterms}[1]{\terms{\Gamma}{#1}}
\newcommand{\ext}[2]{{#1\!\centerdot\! #2}}
\newcommand{\ty}{\ensuremath{\,:\,}}
\newcommand{\pair}[1]{\ensuremath{\langle #1\rangle}}
\newcommand{\exdot}{\ensuremath{\!\centerdot\!}}
\newcommand{\texdot}{\ensuremath{\centerdot}}

% Id types
\newcommand{\Id}{\mathsf{Id}}
\newcommand{\id}[1]{\Id_{#1}}
\newcommand{\refl}{\mathsf{refl}}
\newcommand{\idrec}{\mathsf{idrec}}
\newcommand{\jay}{\mathsf{j}}
\renewcommand{\i}{\mathsf{i}}

% Universe
\newcommand{\U}{\ensuremath{\mathcal{U}}}
\newcommand{\UU}{\ensuremath{\,\dot{\mathcal{U}}}}

\newcommand{\SSet}{\ensuremath{\,\dot{\Set}}}

\newcommand{\V}{\ensuremath{\mathcal{V}}}
\newcommand{\VV}{\ensuremath{\dot{\mathcal{V}}}}

\newcommand{\Fib}{\ensuremath{\mathsf{Fib}}}
\newcommand{\FFib}{\ensuremath{\dot{\mathsf{Fib}}}}
\newcommand{\TFib}{\ensuremath{\mathsf{TFib}}}
\newcommand{\TTFib}{\ensuremath{\dot{\mathsf{TFib}}}}

\newcommand{\elem}[1]{\textstyle\int\!{#1}}


% Adjunction as a two-way rule
\newcommand{\adjunction}[2]{%
  \begin{tabular}{c}
    $#1$ \\
    \noalign{
      \vskip 2pt      
      \hrule
      \vskip 1pt      
      \hrule
      \vskip 2pt      
      }
    $#2$
  \end{tabular}
  }

\newcommand{\adjunctionx}[3]{%
  \begin{tabular}{c}
    $#1$ \\
    \noalign{
      \vskip 2pt      
      \hrule
      \vskip 1pt
      \hrule
      \vskip 2pt      
      }
    $#2$ \\
    \noalign{
      \vskip 2pt      
      \hrule
      \vskip 1pt
      \hrule
      \vskip 2pt      
      }
    $#3$
  \end{tabular}
  }

\newcommand{\adjrule}{\noalign{\vskip 2pt \hrule \vskip 1pt \hrule \vskip 2pt}}
\newcommand{\longadjunction}[1]{
\begin{tabular}{>{$}c<{$}}
#1
\end{tabular}
}

% commutative diagrams
\usepackage{tikz}
\usepackage{pdfpages}
\usepackage{tikz-cd}
\newcommand{\pbmark}{\ar[dr, phantom, "\lrcorner" very near start, shift right=.5ex]}	% pullback mark
\newcommand{\pbbmark}{\ar[drr, phantom, "\lrcorner" very near start, shift right=.5ex]}	% pullback mark



% theorem styles
\newtheorem{theorem}{Theorem}
\newtheorem*{theorem*}{Theorem}
\newtheorem{proposition}[theorem]{Proposition} 
\newtheorem{lemma}[theorem]{Lemma}
\newtheorem{corollary}[theorem]{Corollary} 

\theoremstyle{remark}
\newtheorem{remark}[theorem]{Remark} 
\newtheorem*{remarks*}{Remarks}
\newtheorem{example}[theorem]{Example}

\theoremstyle{definition}
\newtheorem{definition}[theorem]{Definition}

%%%%%%%%%%%%%%%%%%%%%%%%%%%%%%%%%%%%%%%%%%%%%%%%%%%%
\begin{document}
%%%%%%%%%%%%%%%%%%%%%%%%%%%%%%%%%%%%%%%%%%%%%%%%%%%%

\title{A remark on Hofmann-Streicher universes}
\author{Steve Awodey}
\maketitle

\begin{abstract}
We have another look at the construction by Hofmann and Streicher of a semantic universe ${\mathsf{E}l} \to U$ for the interpretation of Martin-L\"of type theory in a presheaf category $\psh{\C}$.  It turns out that ${\mathsf{E}l} \to U$  can be described as the \emph{categorical nerve} of the classifier $\dot{\Set}^{\mathsf{op}} \to \op{\Set}$ for discrete fibrations in $\Cat$, where the nerve is right adjoint to the well-known ``Grothendieck construction'' taking a presheaf $P : \op{\C}\to\Set$ to its category of elements ~$\int_\C P$.
\end{abstract}

%\setcounter{tocdepth}{1}
%\tableofcontents

%%%%%%%%%%%%%%%%%%%%%%%%%%%%%%%%%%%%%%%%%%%%%%%
%\section*{Introduction}
%%%%%%%%%%%%%%%%%%%%%%%%%%%%%%%%%%%%%%%%%%%%%%%

\noindent Let $\widehat{\C} = \psh{\C}$ be the category of presheaves on a small category~$\C$.

%%%%%%%%%%%%%%%%%%%%%%%%%%%%%%%%%%%%%%%%%%%%%%%%%%%%%%%
\subsection*{1. The Hofmann-Streicher universe}\label{sec:U}
%%%%%%%%%%%%%%%%%%%%%%%%%%%%%%%%%%%%%%%%%%

In \cite{HS:1997} the authors define a (type-theoretic) \emph{universe} 
\begin{equation}\label{eq:universe}
E \too U
\end{equation}
 in $\widehat{\C}$ as follows. For $I\in\C$, set
 \begin{align}
	U(I)\ &=\ \mathsf{ob}(\widehat{\C/_I}) \label{eq:universeob}\\ 
	E(I)\ &=\ {\textstyle \coprod_{A\in U(I)}{\mathsf{E}l}(\langle I, A\rangle)}\\
 	{\mathsf{E}l}(\langle I, A\rangle)\ &=\ A(id_I) \label{eq:universeel}
 \end{align}
with an evident associated action on morphisms, which need not concern us for the moment.  A few comments are required: 
%
\begin{enumerate}
%
\item  In \eqref{eq:universeob}, we have taken the underlying set of objects of the category $\widehat{\C/_I}=\psh{\C/_I}$.
% 
\item In \eqref{eq:universeel}, and throughout, the authors steadfastly adopt the ``categories with families'' point of view in describing the morphism $E \ra U$ in $\widehat{\C}$ as an object in
\begin{equation}\label{eq:elements}\textstyle
\widehat{\int_{\C}U}\ \simeq\ \widehat{\C}/_U\,,
\end{equation}
and thus as a presheaf on the \emph{category of elements} $\int_{\C}U$, rather than specifying the object $E$ in $\widehat{\C}$.  Thus the argument $\langle I, A\rangle\in \int_{\C}U$ in \eqref{eq:universeel} consists of an object $I\in\C$ and an element $A\in U(I)$.
%
\item In order to account for size issues, the authors assume a Grothendieck universe $\U$ in $\Set$, the elements of which are called \emph{small}.  The category $\C$ is then assumed to be small, as are the values of the presheaves (unless otherwise stated).  
%
\end{enumerate}

The presheaf $U$, which is not small, is regarded as the Grothendieck universe $\U$ ``lifted'' from $\Set$ to $\psh{\C}$.
We will analyse the construction of \eqref{eq:universe} from a slightly different perspective in order to arrive at its basic property as a classifier for small families in $\widehat\C$. 

%%%%%%%%%%%%%%%%%%%%%%%%%%%%%%%%%%%%%%%%%%
\subsection*{2. An unused adjunction}

For a presheaf $X$ on $\C$, recall that the category of elements is the comma category,
\[\textstyle
\int_\C X\ =\ \yon_\C/_X\,,
\] 
where $\yon_\C : \C \to \psh\C$ is the Yoneda embedding, which we may supress and write simply $\C/_X$. 
While the category of elements $\int_\C X$ is used in the specification of the Hofmann-Streicher universe ${\mathsf{E}l} \ra U$ at the point \eqref{eq:elements}, the authors seem to have missed a trick  which would have simplified things:

\begin{proposition}
The category of elements functor $\int_\C : \widehat\C \too \Cat$ has a right adjoint, which we denote
\[
\nu_\C : \Cat \too \widehat\C\,.
\]
For a small category $\A$, we call the presheaf $\nu_\C(\A)$ the \emph{$\C$-nerve} of $\A$.
\end{proposition}
\begin{proof}
As suggested by the name, the adjunction $\int_\C\! \dashv \nu_\C$ can be seen as the familiar ``realization $\dashv$ nerve'' construction with respect to the covariant functor $\C/- : \C\to\Cat$, as indicated below.
\begin{equation}\label{eq:nerve}\textstyle
\begin{tikzcd}
	 \widehat\C \ar[rr, swap,"\int_\C"] &&  \ar[ll, swap,bend right=20, "{ \nu_\C}"] \Cat\\  
	 \\
	\C \ar[uu, hook, "\yon"] \ar[rruu, swap,"{\C/_{-}}"] &&
 \end{tikzcd}
 \end{equation}
%
In detail, for  $\A\in\Cat$ and $c\in\C$, let $\nu_{\C}(\A)(c)$ be the Hom-set of functors,
\begin{align*}
\nu_\C(\A)(c) &= \Cat\big( {\C/_c}\,,\, \A \big)\,,
\end{align*}
with contravariant action on $h : d\to c$ given by precomposing a functor $P : {\C/_c}\to\A$  with the postcomposition functor
\[
{\C/_h} : {\C/_d}\too {\C/_c} \,.
\]
For the adjunction, observe that the slice category $\C/_c$ is the category of elements of the representable functor $\y{c}$\,,
\[\textstyle
\int_\C\y{c}\ \cong\ \C/_c\,.
\]
 Thus for representables $\y{c}$\,, we have the required natural isomorphism
 \[\textstyle
 \widehat\C\big( \y{c}\,,\, \nu_\C(\A) \big)\ \cong\ \nu_\C(\A)(c)\  =\ \Cat\big( {\C/_c}\,,\, \A \big)\ \cong\ \Cat\big( \int_\C\y{c}\,,\, \A \big)\,.
  \]
For arbitrary presheaves $X$, one uses the presentation of $X$ as a colimit of representables over the index category $\int_\C X$, and the easy to prove fact that $\int_\C$ itself preserves colimits.  Indeed, for any category $\D$, we have an isomorphism in $\Cat$,
\[
\varinjlim_{d\in\D}\,\D/_d \ \cong\ \D\,.
\]
\end{proof}

When $\C$ is fixed, as here, we may omit the subscript from the notation $\yon_\C$ and  $\int_\C$ and $\nu_\C$.  The unit and counit maps of the adjunction $\int \dashv \nu$, vis.\
\begin{align*}\textstyle
\eta :&\ \textstyle  X \too \nu{\elem{X}}\,, \\
\epsilon :&\ \textstyle  \elem\nu\A \too \A\,,
\end{align*}
 are as follows.  At $c\in\C$, for $x : \y{c}\ra X$, the functor $(\eta_X)_c(x) : \C/_c \to \C/_X$ is just composition with $x$, 
\begin{equation}\label{eq:eta}
(\eta_X)_c(x) = \C/_x : \C/_c \too \C/_X\,.
\end{equation}
For $\A\in\Cat$, the functor $ \epsilon : \int\nu\A \to \A$ takes a pair $(c\in\C, f : \C/_c \to \A)$ to the object $f(1_c) \in \A$,
\[
\epsilon(c,f) = f(1_c).
\]
\begin{lemma}\label{lemma:natpb}
For any $f : Y\to X$, the naturality square below is a pullback.
\begin{equation}\label{eq:naturality}\textstyle
\begin{tikzcd}
	 Y \ar[d, swap,"f"] \ar[r, "{\eta_Y}"] & \nu{\int\!{Y}} \ar[d, "{ \nu{\int\!{f}}}"]\\  
	X \ar[r, swap,"{\eta_X}"] &   \nu{\int\!{X}}.
 \end{tikzcd}
 \end{equation}
\end{lemma}

\begin{proof}
It suffices to prove it for the case $f : X\ra 1$.  Thus consider the square 
\begin{equation}\label{eq:naturalityobject}\textstyle
\begin{tikzcd}
	 X \ar[d] \ar[r, "{\eta_X}"] & \nu{\int\!{X}} \ar[d]\\  
	1\ar[r, swap,"{\eta_1}"] &   \nu{\int\! 1}.
 \end{tikzcd}
 \end{equation}
Evaluating at $c\in\C$ and applying \eqref{eq:eta} then gives the following square in $\Set$.
\begin{equation}\label{eq:naturalityobjecteval}\textstyle
\begin{tikzcd}
	 Xc \ar[d] \ar[r, "{\C/_{-}}"] & \Cat\big( {\C/_c}\,,\, {\C/_X}\, \big) \ar[d]\\  
	1c\ar[r, swap, "{\C/_{-}}"] &   \Cat\big( {\C/_c}\,,\, \C/_1 \big)
 \end{tikzcd}
 \end{equation}
The image of $*\in 1c$ along the bottom is the forgetful functor $U_c : \C/_c\to \C$, and its fiber under the map on the right is therefore the set of functors $F : {\C/_c}\to {\C/_X}$ such that $U_X\circ F = U_c$, where $U_X : \C/_X\to \C$ is also a forgetful functor. But any such $F$ is easily seen to be uniquely of the form $\C/_{x}$ for $x = F(1_c) : \y{c} \to X$.
\end{proof}

%%%%%%%%%%%%%%%%%%%%%%%%%%%%%%%%%%%%%%%%%%
\subsection*{3. Classifying families}

For the terminal presheaf $1\in\widehat{\C}$, we have $\elem{1} \cong\C$, so for every $X\in\widehat{\C}$ there is a canonical projection  $\elem X \ra\C$, which is easily seen to be a discrete fibration.  It follows that for any map $Y\to X$ of presheaves, the associated map $\elem Y \to \elem X$ is also a discrete fibration. 
Ignoring size issues for the moment, recall that discrete fibrations in $\Cat$ are classified by the forgetful functor $\op{\dot{\Set}}\to \op{\Set}$ from (the opposites of) the category of pointed sets to that of sets.  For every presheaf $X\in\widehat{\C}$, we therefore have a pullback diagram in $\Cat$,
\begin{equation}\label{eq:classifyuniversecat}\textstyle
\begin{tikzcd}
	 \elem X \ar[d] \ar[r] \pbmark & \op{\dot{\Set}} \ar[d]\\  
	\C \ar[r,swap,"X"] &  \op{\Set}.
 \end{tikzcd}
 \end{equation}
 Transposing by the adjunction $\int\dashv \nu$ then gives a commutative square in $\widehat{\C}$,
\begin{equation}\label{eq:classifyuniversetype}\textstyle
\begin{tikzcd}
	 X \ar[d] \ar[r] & \nu\op{\dot{\Set}} \ar[d]\\  
	1 \ar[r,swap,"\tilde{X}"] &  \nu\op{\Set}.
 \end{tikzcd}
 \end{equation}

\begin{lemma}
The square \eqref{eq:classifyuniversetype} is a pullback in $\widehat{\C}$. More generally, for any map $Y\ra X$ in $\widehat{\C}$, there is a pullback square 
\begin{equation}\label{eq:classifyuniversefamily}\textstyle
\begin{tikzcd}
	 Y \ar[d] \pbmark \ar[r] & \nu\op{\dot{\Set}} \ar[d] \\  
	X \ar[r] &  \nu\op{\Set}\,.
 \end{tikzcd}
 \end{equation}
\end{lemma}

\begin{proof}
Apply the right adjoint $\nu$ to the pullback square \eqref{eq:classifyuniversecat} and paste the naturality square \eqref{eq:naturality} from Lemma \ref{lemma:natpb} on the left, to obtain the transposed square \eqref{eq:classifyuniversefamily} as a pasting of two pullbacks.
\end{proof}

Let us write $\VV \to \V$ for the vertical map on the right in \eqref{eq:classifyuniversefamily}, so that
\begin{align}\label{eq:universedef}\textstyle
\VV\, &=\, \nu\op{\dot{\Set}}\\  
\V\, &=\, \nu\op{\Set}.\notag
 \end{align}
 
 We can summarize our results so far as follows.

 \begin{proposition}\label{prop:Vclassifies}
The nerve $\VV\to\V$  of the classifier for discrete fibrations $\op\SSet\to\op\Set$, as defined in \eqref{eq:universedef}, classifies natural transformations $Y\to X$ in $\widehat{\C}$, in the sense that there is always a pullback square,
\begin{equation}\label{eq:classifyuniversefamily2}\textstyle
\begin{tikzcd}
	 Y \ar[d] \pbmark \ar[r] & \VV \ar[d] \\  
	X \ar[r,swap, "\tilde{Y} "] &  \V.
 \end{tikzcd}
 \end{equation}
The classifying map $\tilde{Y} : X\to \V$ is determined by the adjunction $\int \dashv \nu$ as the transpose of the classifiyng map of the discrete fibration $\elem X\to\elem Y$.
\end{proposition}
 
 Of course, $\VV\to\V$ itself cannot be a map in $\widehat{\C}$, for reasons of size.

%%%%%%%%%%%%%%%%%%%%%%%%%%%%%%%%%%%%%%%%%%
\subsection*{4. Small maps}

Let $\alpha$ be a cardinal number and call the sets that are strictly smaller $\alpha$-\emph{small}.  Let $\Set_\alpha\hook\Set$ be the full subcategory of $\alpha$-small sets.  
Call a presheaf $X : \op{\C} \to \Set$ $\alpha$-small if all of its values are $\alpha$-small sets, and thus if, and only if, it factors through $\Set_\alpha\hook\Set$. Call a map $f:Y\to X$ of presheaves $\alpha$-small if all of the fibers $f_c^{-1}\{ x\} \subseteq Yc$ are $\alpha$-small sets (for all $c\in\C$ and $x\in Xc$). The latter condition is of course equivalent to saying that, in the pullback square over the element $x:\y{c} \to X$, 
\begin{equation}\label{eq:smallmap}\textstyle
\begin{tikzcd}
	 Y_x \ar[d] \pbmark \ar[r] & Y \ar[d, "f"] \\  
	\y{c} \ar[r,swap,"x"] &  X,
 \end{tikzcd}
 \end{equation}
the presheaf $Y_x$ is $\alpha$-small.

% Consider $\Set_\alpha^{\C^{\mathrm{op}}}$, the category of $\alpha$-small set valued presheaves on $\C$.  
% and similarly for the small pointed sets $\SSet_\alpha\hook\SSet$

Now let us restrict the specification \eqref{eq:universedef} of $\VV\to\V$ to the $\alpha$-small sets:
\begin{align}\label{eq:universedefalpha}\textstyle
\VV_\alpha\, &=\, \nu \dot{\Set^{\mathsf{op}}_\alpha}\\  
\V_\alpha\, &=\, \nu \Set^{\mathsf{op}}_\alpha. \notag
 \end{align}
Then the evident forgetful map $\VV_\alpha\to\V_\alpha$ \emph{is} a map in the category $\widehat{\C}$ of presheaves, and it is in fact $\alpha$-small. Moreover, it has the following basic property, which is just a restriction of the basic property of $\VV\to\V$ stated in Proposition \ref{prop:Vclassifies}.

 \begin{proposition}\label{prop:Valphaclassifies}
The map $\VV_\alpha\to\V_\alpha$ classifies $\alpha$-small maps $f:Y\to X$ in $\widehat{\C}$, in the sense that there is always a pullback square,
\begin{equation}\label{eq:classifyuniversefamilyalpha}\textstyle
\begin{tikzcd}
	 Y \ar[d] \pbmark \ar[r] & \VV_\alpha \ar[d] \\  
	X \ar[r,swap, "\tilde{Y}"] &  \V_\alpha.
 \end{tikzcd}
 \end{equation}
The classifying map $\tilde{Y} : X\to \V_\alpha$ is determined by the adjunction $\int \dashv \nu$ as (the factorization of) the transpose of the classifiyng map of the discrete fibration $\elem X\to\elem Y$. 
\end{proposition}

\begin{proof} If $Y\to X$ is $\alpha$-small, its classifying map $\tilde{Y} : X\to\V$ factors through $\V_\alpha \hook \V$, as indicated below, 
\begin{equation}\label{eq:classifyuniversetype2}\textstyle
\begin{tikzcd}
	 Y \ar[d] \ar[rr, bend left] \ar[r] & \nu\op{\dot{\Set_\alpha}} \ar[d] \ar[r,hook] & \nu\op{\dot{\Set}} \ar[d]\\  
	X \ar[rr, bend right, swap,"\tilde{Y}"] \ar[r] &  \nu\op{\Set_\alpha} \ar[r,hook] &  \nu\op{\Set},
 \end{tikzcd}
 \end{equation}
%
in virtue of the following adjoint transposition,
\begin{equation}\label{eq:adjointtranspose}\textstyle
\begin{tikzcd}
	 \elem Y \ar[d] \ar[rr, bend left] \ar[r] & \op{\dot{\Set_\alpha}} \ar[d] \ar[r,hook] & \op{\dot{\Set}} \ar[d]\\  
	 \elem X \ar[rr, bend right, ] \ar[r]  &  \op{\Set_\alpha} \ar[r,hook]  &  \op{\Set}.
 \end{tikzcd}
  \end{equation}
Note that the square on the right is evidently a pullback, and the one on the left therefore is, too, because the outer rectangle is the classifying pulback of the discrete fibration $\elem Y \to \elem X$, as stated.  Thus the left square in \eqref{eq:classifyuniversetype2} is a pullback.
\end{proof}

%%%%%%%%%%%%%%%%%%%%%%%%%%%%%%%%%%%%%%%%%%
\subsection*{5. Examples}

\begin{enumerate}
%
\item Let $\alpha = \kappa$ a strongly inaccessible cadinal, so that $\mathsf{ob}({\Set_\kappa})$ is a Grothendieck universe.  Then the Hofmann-Streicher universe of \eqref{eq:universe} is recovered in the present setting as the $\kappa$-small map classifier
\begin{equation*}
E\, \cong\, \VV_\kappa \too \V_\kappa\, \cong\, U
\end{equation*}
 in the sense of Proposition \ref{prop:Valphaclassifies}.  Indeed, for $c\in\C$, we have 
 \begin{align}
  \V_{\kappa}{c}\ &=\ \nu(\Set^{\mathsf{op}}_\kappa)(c) = \Cat\big( {\C/_c}\,,\, \Set^{\mathsf{op}}_\kappa \big)\  =\ \mathsf{ob}(\widehat{\C/_c})\ =\ U{c} \,.
   \end{align} 
For $\VV_{\kappa}$ we then have,
   \begin{align}\label{eq:veedotc}
   \VV_{\kappa}{c}\ =\ \nu(\SSet^{\mathsf{op}}_\kappa)(c)\ &=\ \Cat\big( {\C/_c}\,,\, \SSet^{\mathsf{op}}_\kappa \big) \notag \\ 
   &\cong\ {\textstyle \coprod_{A\in\V_{\kappa}{c}}\Cat_{{\C/_c}}\big( {\C/_c}\,,\, A^*\Set^{\mathsf{op}}_\kappa \big)}
   \end{align}
   where the $A$-summand in \eqref{eq:veedotc} is defined by taking sections of the  pullback indicated below.
   \begin{equation}\label{eq:pbforindexing}\textstyle
\begin{tikzcd}
	A^*\Set^{\mathsf{op}}_\kappa \ar[d] \ar[r] \pbmark & \SSet^{\mathsf{op}}_\kappa \ar[d]\\  
	\C/_c \ar[r,swap,"A"] \ar[u, bend left, dotted] \ar[ur, dotted] &  \Set^{\mathsf{op}}_\kappa
 \end{tikzcd}
 \end{equation}
 But $A^*\Set^{\mathsf{op}}_\kappa\ \cong\ {\textstyle \int_{\C/_c}\!A}$ over $\C/_c\,$, and sections of this discrete fibration in $\Cat$ correspond uniquely to natural maps $1\to A$ in $\widehat{{\C/_c}}$.  Since $1$  is representable in $\widehat{{\C/_c}}$ we can continue \eqref{eq:veedotc} by
  \begin{align*}
   \VV_{\kappa}{c}\ &\cong\ {\textstyle \coprod_{A\in \V_{\kappa}{c}}\Cat_{{\C/_c}}\big( {\C/_c}\,,\, A^*\Set^{\mathsf{op}}_\kappa \big)}\\
   	&\cong\ {\textstyle \coprod_{A\in \V_{\kappa}{c}} \widehat{{\C/_c}}(1, A)}\\
	&\cong\ {\textstyle \coprod_{A\in \V_{\kappa}{c}} A(1_c) } \\
	& =\ {\textstyle \coprod_{A\in \V_{\kappa}{c}} {\mathsf{E}l}(\langle c, A\rangle)}\\
	& =\  E c\,.
   \end{align*}
 
% The second line is in virtue of the iso,
%  \begin{align}
%  \V_{\kappa}{c}\ &=\ \nu(\Set^{\mathsf{op}}_\kappa)(c) = \Cat\big( {\C/_c}\,,\, \Set^{\mathsf{op}}_\kappa \big)\  =\ U{c} \,\notag\\ 
%   \VV_{\kappa}{c}\ &=\ \nu(\SSet^{\mathsf{op}}_\kappa)(c) = \Cat\big( {\C/_c}\,,\, \SSet^{\mathsf{op}}_\kappa \big)
%   = \textstyle{ \coprod_{A\in U{c}} }A(1_c) = \textstyle\coprod_{A\in U{c}}{\mathsf{E}l}(\langle c, A\rangle) \,.
% \end{align}

\item By functoriality of $\nu : \Cat \to \widehat{\C}$, a sequence of Grothendieck universes $$\U \subseteq {\U\,}' \subseteq ...$$ in $\Set$ gives rise to a (cumulative) sequence of type-theoretic universes $$\V \mono {\V\,}' \mono ...$$ in $\widehat{\C}$. More precisely, there is a sequence of  cartesian squares,
\begin{equation}\label{eq:Vhierarchy}\textstyle
\begin{tikzcd}
	 \VV \ar[d] \ar[r,tail] \pbmark & {\VV\,'} \ar[d] \ar[r,tail] \pbmark & \dots \\  
	 \V  \ar[r, tail]  &  {\V\,}' \ar[r,tail]  & \dots\,,
 \end{tikzcd}
  \end{equation}
in the image of $\nu : \Cat\too\widehat\C$, classifying small maps in $\widehat\C$ of increasing size, in the sense of Proposition \ref{prop:Valphaclassifies}.

\item Let $\alpha = 2$ so that $1\to 2$ is the subobject classifier of $\Set$, and 
$$\mathbbm{1} = \SSet^{\mathsf{op}}_2 \too  \Set^{\mathsf{op}}_2 = \mathbbm{2}$$ is then a classifier in $\Cat$ for full subcategories $\mathbb{S}\hook\A$ that are closed under the domains of arrows $a\to s$ for $s\in\mathbb{S}$ (``total sieves'').  The
%\[
%\VV_{2} = \nu \mathbbm{1} = 1 \too \Omega = \nu \mathbbm{2} = \V_{2} \,,
%\] 
 ``lifted universe''  $\VV_{2}  \to \V_{2}$ is then the subobject classifier $1\to\Omega$ of $\widehat\C$,
\[
1 = \nu \mathbbm{1} = \VV_{2} \too  \V_{2} = \nu \mathbbm{2} = \Omega  \,.
\]

\item Let $i : \mathbbm{2} \hook \Set_\kappa$ and $ p : \Set_\kappa \to \mathbbm{2} $ be the embedding-retraction pair with $i : \mathbbm{2} \hook \Set_\kappa$ the inclusion of the full subcategory on the sets $ \{0, 1\}$ and $p : \Set_\kappa \to \mathbbm{2}$ the retraction that takes $0 = \emptyset$ to itself, and everything else (i.e.\ the non-empty sets) to $1 = \{\emptyset\}$. There is a retraction (of arrows) in $\Cat$,
%
\begin{equation}\label{eq:Setretraction}\textstyle
\begin{tikzcd}
	 \mathbbm{1} \ar[d]  \ar[r,hook] \pbmark & \SSet_\kappa \ar[d] \ar[r] &\mathbbm{1} \ar[d] \\  
	 \mathbbm{2}   \ar[r,hook,swap, "i"]  &  \Set_\kappa \ar[r, swap, two heads, "p"]  & \mathbbm{2} 
 \end{tikzcd}
  \end{equation}
  where the left square is a pullback.  
    
By the functoriality of $\nu : \Cat \to \widehat{\C}$ we then have a retract diagram in $\widehat\C$, again with a pullback on the left, 
\begin{equation}\label{eq:Vretraction}\textstyle
\begin{tikzcd}
	 1 \ar[d]  \ar[r,hook] \pbmark & \VV_\kappa \ar[d] \ar[r] &1\ar[d] \\  
	 \Omega   \ar[r,hook,swap, "\{-\}"]  &  \V_\kappa  \ar[r, swap, two heads, "{[-]}"]  &  \Omega 
 \end{tikzcd}
  \end{equation}
  where for any $\phi : X\to \Omega$ the subobject $\{\phi\}\mono X$ is classified as a small map by the composite $\{\phi\} : X\to \V_\kappa$\,, and for any small map $A\to X$, the image $[A] \mono X$ is classified as a subobject by the composite $[\alpha] : X\to \V_\kappa \to \Omega$\,, where $\alpha : X\to \V_\kappa$ classifies $A\to X$.  The idempotent composite $$|\!|\!-\!|\!| = \{[-]\} : \V_\kappa  \too \V_\kappa $$ is the \emph{propositional truncation modality} in the natural model of type theory given by $\VV_\kappa\to \V_\kappa$ (see \cite{AGH}).
\end{enumerate}

%%%%%%%%%%%%%%%%%%%%%%%%%%%%%%%%%%%%%%%%%%
\subsection*{Acknowledgement}

Thanks to Mathieu Anel and Emily Riehl for discussions.

\bibliographystyle{alpha}
\bibliography{../references}

%%%
\end{document}
%%%



By a \emph{small fibration} we mean a fibration in the category of small cubical sets, which we identify with the evident subcategory $\cSet_\kappa \subseteq \cSet$.  Finally, let $\dot{\Set}_\kappa$ be the category of small pointed sets, i.e.\ the coslice category $1/\Set_\kappa$.  There is an evident forgetful functor  $U:\dot{\Set}_\kappa \ra {\Set}_\kappa$.
% which induces another one $U^{\C^{op}} : \dot{\Set}_\kappa^{\C^{op}} \ra {\Set}_\kappa^{\C^{op}}$ from small pointed presheaves to small presheaves.

\begin{definition}
The ($\kappa$-)\emph{universe} $p : \VV \ra \V$ in $\cSet$ is defined:
\begin{enumerate}

\item $\V_n = \{ A: \C/[n] \ra \Set^{op}_\kappa\ \}$, the \emph{set} of small presheaves on $\C/[n]$.  

The action of a map $h : [m] \ra [n]$ in  $\C$  is given by precomposition with postcomposition: from  $h : [m] \ra [n]$ we have  $\C/h : \C/[m] \ra \C/[n]$, which we precompose with any $A: \C/[n]\ra \Set_\kappa^{op} $ to get $A.h = A\circ \C/h$,
\begin{equation}\label{diagram:U1}
\xymatrix{
[n] & \C/[n] \ar[r]^{A} & \Set_\kappa^{op} \\
[m] \ar[u]^{h} & \C/[m] \ar[u]^{\C/h}  \ar[ru]_{A.h} &
}
\end{equation}

\item $\VV_n = \{ a: \C/[n] \ra \dot{\Set}^{op}_\kappa\ \}$, the \emph{set} of small pointed presheaves on $\C/[n]$, with the corresponding action.

\item For $a \in \VV_n$, let $p_n (a) = U(a) \in \V_n$, where $U : \dot{\Set_\kappa} \ra {\Set_\kappa}$.
\end{enumerate}
\end{definition}
Functoriality of $\V$ and $\VV$ and naturality of $p: \VV\ra\V$ are immediate. 

\begin{lemma}
For each $A : \I^n \ra \V$ there is a canonical choice of a small family $p_A : E_A\ra \I^n$ and a map $q_A: E_A \ra\VV$ making a pullback square as follows.
\begin{equation}\label{diagram:U2}
\xymatrix{
E_A \ar[d]_{p_A} \ar[r]^{q_A}  \pbcorner& \VV\ar[d]^{p}  \\
\I^n \ar[r]_{A} & \V
}
\end{equation}
\end{lemma}
%
\begin{proof}
Since $\I^n \cong y[n]$ is representable, there is a distinguished associated presheaf $A : (\C/[n])^{op} \ra \Set_\kappa$. Define $p_A : E_A \ra \I^n$ by
\[
(E_A)_k = \coprod_{h\in \C(k,n)} A(h) \qquad \ni (h, a)\,
\]
with first projection $(p_A)_k(h,a) = h$. Note that $(E_A)_k$ is small.
Then let $q_A:E_A \ra\VV$ be defined on $(h,a): \I^k \ra E_A$ by 
\[
(q_A)\circ(h, a) = a \in Ah
\]
as illustrated below.
\begin{equation}\label{diagram:U3}
\xymatrix@=5em{
& E_A \ar[d]^>>>>>>{p_A} \ar@{..>}[r]^{q_A} & \VV\ar[d]^{p}  \\
\I^k \ar[r]_{h} \ar[ru]^{(h,a)}\ar[rru]_>>>>>>>>>>>{a} & \I^n \ar[r]_{A} & \V
}
\end{equation}
The proof that the square is a pullback is left to the reader.
\end{proof}

\begin{lemma}\label{charforfamovercube}
For each small family $p_E : E \ra \I^n$ there is a canonical map $\chi_E : \I^n \ra \V$ and a map $q_E:E \ra\VV$ making a pullback square as follows.
\begin{equation}\label{diagram:U2}
\xymatrix{
E \ar[d]_{p_E} \ar[r]^{q_E}  \pbcorner& \VV\ar[d]^{p}  \\
\I^n \ar[r]_{\chi_E} & \V
}
\end{equation}
\end{lemma}
%
\begin{proof}
It suffices to give a small set $(\chi_E)_k(h)$ for each $h :[k] \ra [n]$ in a way that is functorial in $h \in \C/[n]$ and natural in $[k]$.  Thus let
\[
(\chi_E)_k(h) := \Gamma(h, E)\ = \ \{ e : \I^k \ra E \ |\ p_E\circ e = h \}.
\]
\[
\xymatrix{& E \ar[d]^{p_E} \\ \I^k \ar@{..>}[ru]^{e}\ar[r]_h & \I^n } 
\]
which is small if each $E_k$ is.  

To define $q_E:E \ra\VV$,  take any $e : \I^k \ra E$ and first compose with $p_E$ and observe that $e \in  \Gamma(p_Ee, E)$.  Thus the assignment gives a map $\dot{e} : \I^k \ra \VV$ making the solid arrows in the following commute.
\begin{equation}\label{diagram:U4}
\xymatrix@=5em{
&E \ar[d]^>>>>>>>{p_E} \ar@{..>}[r]^{q_E} & \VV\ar[d]^{p}  \\
\I^k \ar[r]_{p_Ee} \ar[ru]^{e}\ar[rru]_>>>>>>>>>>>{\dot{e}} & \I^n \ar[r]_{E} & \V
}
\end{equation}
Since the assignment of $\dot{e}$ to $e$ is natural in $[k]$, we get the required map $q_E : E\ra \VV$.
The proof that the square is a pullback is again left to the reader.
\end{proof}

\begin{corollary}
Given a small family $p_E : E \ra \I^n$ there is a unique isomorphism $E \cong E_{\chi_E}$ over $\I^n$ making a commutative diagram as follows.

\begin{equation}\label{diagram:U5}
\xymatrix@=5em{
E \ar[rd]_{p_E} \ar@/^2.0pc/[rr]^{q_E} \ar[r]^{\cong}  & E_{\chi_E} \ar[d]^{p_{\chi_E}} \ar[r]_{q_{\chi_E}} & \VV \ar[d]^{p}  \\
 &\I^n \ar[r]_{\chi_E} & \V
}
\end{equation}
\end{corollary}
%

\begin{proposition}\label{prop:familyclassifier}
For any cubical set $X$ and any small family $p_E : E \ra X$ there are canonical maps $\chi_E : X \ra \V$ and $q_E : E \ra\VV$ making a pullback square as follows.
\begin{equation}\label{diagram:U6}
\xymatrix{
E \ar[d]_{p_E} \ar[r]^{q_E}  \pbcorner & \VV\ar[d]^{p}  \\
X \ar[r]_{\chi_E} & \V
}
\end{equation}
Moreover, $\chi_E$ and $q_E$ are uniquely determined by the equations \eqref{eqns:U} below.
\end{proposition}

\begin{proof}
Write $X = \varinjlim_{x} \I^n$ as a colimit of a cocone of maps $ x : \I^n \ra X$ from representables, over the canonical index category $([n],x) \in \int_\C X$.  Form the family of pullback squares below, where the arrows with a dot represent cocones, and the cocone consisting of the $q_x : E_x \ra E$  is determined by taking pullbacks along $p_E$, and is therefore also a colimit. 
\begin{equation}\label{diagram:U7}
\xymatrix@=3em{
E_x \ar[d]_{p_{E_x}} \ar[r]_{\cdot}^{q_x} \ar@/^2.0pc/[rr]^{q_{E_x}} \pbcorner & E \ar[d]_{p_{E}} \ar@{..>}[r]^{q_E} & \VV\ar[d]^{p}  \\
\I^n \ar@/_2.0pc/[rr]_{\chi_{E_x}}  \ar[r]^{\cdot}_x & X \ar@{..>}[r]_{\chi_E} & \V
}
\end{equation}
The maps $\chi_{E_x}$  and $q_{E_x}$ are determined by lemma \ref{charforfamovercube}, since the families $p_{E_x}$ are small if $p_E: E \ra X$ is.  Thus we can define the indicated maps $\chi_E$ and $q_E$ from the colimits as those uniquely determined by the equations:
\begin{align}\label{eqns:U}
\chi_E \circ x &= \chi_{E_x}\\
q_E \circ q_{x} &= q_{E_x}
\end{align}
The square on the right is a pullback because the outer squares are all pullbacks,  the family of left-hand squares are pullbacks, and the family of maps $x : \I^n \ra X$ covers $X$.
\end{proof}

\begin{remark} 
Note that the classification operation $$\chi : \cSet_\kappa/X \ra \cSet(X, \V)$$ again has the evident ``pullback of $p : \VV\ra\V$'' operation 
$$E : \cSet(X, \V) \ra \cSet_\kappa/X $$ as a left (quasi-)inverse $
E \cong E_{\chi_E}$, which is (pseudo-)natural in $X$. 
But there is no corresponding uniqueness of classifying maps, relating $A : X\ra \V$ and $\chi_{E_A} : X\ra\V$.  (This is what is provided, in a suitable sense, by the \emph{univalence} of the universe $\UU\ra\U$ of fibrations, to be established in section \ref{sec:EEP}.)
\end{remark}

%%%%%%%%%%%%%%%%%%%%%%%%%%%%%%%%%%%%%%%%%%
\subsection*{Acknowledgement}

Thanks to Mathieu Anel and Emily Riehl for discussions.

\bibliographystyle{alpha}
\bibliography{../references}

%%%
\end{document}
%%%

